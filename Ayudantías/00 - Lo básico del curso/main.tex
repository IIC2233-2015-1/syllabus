% Copyright 2004 by Till Tantau <tantau@users.sourceforge.net>.
%
% In principle, this file can be redistributed and/or modified under
% the terms of the GNU Public License, version 2.
%
% However, this file is supposed to be a template to be modified
% for your own needs. For this reason, if you use this file as a
% template and not specifically distribute it as part of a another
% package/program, I grant the extra permission to freely copy and
% modify this file as you see fit and even to delete this copyright
% notice. 

% --------------------------------------------
% Template por Patricio López Juri (pelopez2@uc.cl)
\documentclass[spanish]{beamer}


%% -----------------------------------------------------
%% Python & código -------------------------------------
%% -----------------------------------------------------

% Source: https://github.com/bamos/beamer-snippets

\usepackage[T1]{fontenc}
\usepackage{parskip,graphics,tikz,multimedia,hyperref,ulem,multicol}
\usetikzlibrary{arrows,positioning,shapes,decorations.pathmorphing,snakes}
\setlength{\itemsep}{0pt}\setlength{\parskip}{0pt}\setlength{\parsep}{0pt}
\graphicspath{{./images/}}

\usepackage{listings,textcomp,color}
\lstset{language=Python,upquote=true,
  basicstyle=\ttfamily\tiny,numbers=left,
  numberstyle=\tiny,stepnumber=1,numbersep=5pt,
  backgroundcolor=\color{white},frame=single,tabsize=2,
  showspaces=false,showstringspaces=false,showtabs=false,
  breaklines=true,breakatwhitespace=true,escapeinside=||,
  keywordstyle=\color{blue!70},stringstyle=\color{green!70!black!70},
  commentstyle=\color{black!80}\it
}

\usebackgroundtemplate{
  \tikz[overlay,remember picture]
  \node[yshift=10mm,anchor=south east,inner sep=0pt]
    at (current page.south east) {
    % Add small logo here if desired.
    % \includegraphics[width=0.5in]{Images/python-logo.png}
  };
}

\tikzset{
  yn/.style={draw,thick,rounded corners,fill=yellow!20,inner sep=.3cm},
  bn/.style={draw,thick,rounded corners,fill=blue!05,inner sep=.3cm},
  on/.style={draw,thick,rounded corners,fill=orange!20,inner sep=.3cm},
  rn/.style={draw,thick,rounded corners,fill=red!20,inner sep=.3cm},
  greenn/.style={draw,thick,rounded corners,fill=green!20,inner sep=.3cm},
  grayn/.style={draw,thick,rounded corners,fill=gray!20,inner sep=.3cm},
  to/.style={
    ->,>=stealth',shorten >=1pt,semithick,font=\sffamily\footnotesize
  },
  from/.style={
    <-,>=stealth',shorten >=1pt,semithick,font=\sffamily\footnotesize
  },
  tofrom/.style={
    <->,>=stealth',shorten >=1pt,semithick,font=\sffamily\footnotesize
  },
  every node/.style={align=center},
  squig/.style={->,line join=round,decorate, decoration={zigzag,
    segment length=8,amplitude=2,post=lineto,post length=2pt}}
}

\newcommand{\uncheckedBox}{\ensuremath{\square}}
\newcommand{\checkedBox}{\ensuremath{\text{\rlap{\checkmark}}\square}}

\beamertemplatenavigationsymbolsempty

\expandafter\def\expandafter\insertshorttitle\expandafter{%
  \insertshorttitle\hfill%
  \insertframenumber\,/\,\inserttotalframenumber}

\usepackage{enumitem}
\setlist[1]{itemsep=5pt}
\setitemize{label=\usebeamerfont*{itemize item}%
  \usebeamercolor[fg]{itemize item}
  \usebeamertemplate{itemize item}}
  
  
%% -----------------------------------------------------
%% Lenguaje --------------------------------------------
%% -----------------------------------------------------

% \usepackage[T1]{fontenc} Está mas arriba.
\usepackage{selinput}
\SelectInputMappings{%
  aacute={á},
  ntilde={ñ},
  Euro={€}
}
\usepackage[spanish]{babel}


%% -----------------------------------------------------
%% Tema ------------------------------------------------
%% -----------------------------------------------------

% Colores principales
\usetheme{CambridgeUS}

% Para las URL
\definecolor{links}{HTML}{2A1B81}
\hypersetup{colorlinks,linkcolor=,urlcolor=links}


%% -----------------------------------------------------
%% Main ------------------------------------------------
%% -----------------------------------------------------

% Título, debe ir obligatoriamente
\title{Lo básico del curso}

% Subtítulo opcional
\subtitle{Ayudantía 0}

% Ayudantes
\author{Jaime Castro \and Patricio López}

% Universidad
\institute[UC]
{
  Departmento de Ciencia de la Computación\\
  Pontificia Universidad Católica de Chile
}

% Dimensión tiempo espacio a mostrar.
\date{IIC2233, 2015-1}

% Metadada
\subject{Ayudantía de Programación Avanzada}


% If you have a file called "university-logo-filename.xxx", where xxx
% is a graphic format that can be processed by latex or pdflatex,
% resp., then you can add a logo as follows:

% \pgfdeclareimage[height=0.5cm]{university-logo}{puc.png}
% \logo{\pgfuseimage{university-logo}}

% Delete this, if you do not want the table of contents to pop up at
% the beginning of each subsection:
% \AtBeginSubsection[]
% {
%   \begin{frame}<beamer>{Outline}
%     \tableofcontents[currentsection,currentsubsection]
%   \end{frame}
% }


%% -----------------------------------------------------
%% Incio del documento ---------------------------------
%% -----------------------------------------------------

\begin{document}

% Agregamos la página de inicio (la del título)
\begin{frame}
  \titlepage
\end{frame}

% Tabla de contenidos, tiene hipervínculos.
% Los índices son tomados del nombre que se le asigne a las \section.
% Los sub-índices provienen de las \subsections
\begin{frame}{Tabla de contenidos}
  \tableofcontents
  % You might wish to add the option [pausesections]
\end{frame}

% Inicio de sección de Patricio

\section{Sobre el curso}

\subsection{Python}

\begin{frame}{Python}{¿Qué sabemos?}
    \begin{itemize}
        \item Es \alert{\textit{Open Source}}, es decir, su código es público y cualquiera puede verlo.
        \pause
        \item Es \alert{Agnóstico de Sistema Operativo \textit{(OS)}}, o sea, no importa si es Windows, UNIX, etc.
        \pause
        \item Es un lenguaje \alert{interpretado}.
            \pause
            \begin{itemize}
                \item Requiere de un intérprete, eso es lo que instalaron cuando les pedimos que bajaran Python.
                \item Al contrario de otros lenguajes que son \alert{compilados}.
                \item Los intérpretes están escritos en otro lenguaje, distinto a Python. 
            \end{itemize}
    \end{itemize}
\end{frame}

\subsection{PEP8}
\begin{frame}{PEP8}{Standard}

    \begin{exampleblock}{}
      {\large ``Code is read much more often than it is written.''}
      \vskip5mm
      \hspace*\fill{\small--- Guido van Rossum}
    \end{exampleblock}
    
    \pause
    \begin{block}{PEP8}
        Guía de estilo para mantener la consistencia y leíbilidad. \\
        \url{https://www.python.org/dev/peps/pep-0008/}
    \end{block}
    
    \begin{example}
    ¿Espacios vs Tabs? \\
    Se deben usar 4 espacios para identar.
    \end{example}
\end{frame}

\subsection{Repositorio del curso}
\begin{frame}{Sobre el curso}{Repositorio del curso}
    \begin{itemize}
        \item \url{https://github.com/IIC2233-2015-1/syllabus}
        \pause
        \begin{itemize}
            \pause
            \item Se subirá todo el \alert{material} de cátedra y ayudantía.
            \pause
            \item Está el \alert{programa} del curso.
            \pause
            \item Hay una \alert{wiki} con información.
            \pause
            \item Se utilizará como \alert{foro} para preguntas.
        \end{itemize}
    \end{itemize}
    
    \pause
    \begin{block}{Tarea para la casa (o ahora mismo)}
      Crear una cuenta \alert{Github} para poder preguntar en el foro.
  \end{block}
\end{frame}

\begin{frame}{Sobre el curso}{Material del semestre pasado}
    \begin{itemize}
        \item Este semestre comenzó a hacerce el curso en Python,
        \item Todo material viejo les sirve conceptualmente.
        \item El curso es ``Programación Avanzada'', independiente del lenguaje. Lo importante es pensar como programador y usar Google para cualquier duda sobre el lenguaje.
    \end{itemize}
\end{frame}

\section{Git}
\subsection{Introducción}

\begin{frame}{Introducción a Git}{¿Para qué?}
    \begin{itemize}
        \item ¿Cómo revertimos un cambio en el código?
        \pause
        \item ¿Cómo desarrollo una nueva funcionalidad sin arruinar las demás?
        \pause
        \item Si mucha gente trabaja en un mismo código:
            \pause
          \begin{itemize}
            \item ¿Cómo evitamos colisiones en el código?
            \item ¿Cómo sabemos qué fue modificado y quién lo hizo?
        \end{itemize}
    \end{itemize}
\end{frame}

\begin{frame}{Introducción a Git}{¡Habemus Git!}
    \begin{center}
        \includegraphics[height=2cm]{Images/git-logo.png}
    \end{center}
    
    \pause
    \begin{definition}
        Sistema de control de versiones distribuido y Open Source que permite flujos de trabajo \alert{no-lineales}.
  \end{definition}
\end{frame}
    
\begin{frame}{Introducción a Git}{Proveedores}
    Algunos proveedores de este servicio.
    \begin{itemize}
        \item \alert{Github}, el más conocido y el que probablemente usemos.
        \item \alert{Bitbucket}, otra gran alternativa. 
        \pause
        \item Puedes armar tu propio servidor git.
    \end{itemize}
\end{frame}    

\subsection{Uso}
\begin{frame}{Uso}{Glosario I}
    \begin{itemize}
        \item \alert{Branch} o \alert{"rama}:
            \begin{itemize}
            \item Todo repo tiene una rama principal, por lo general: \alert{master}.
            \item Uno crea ramas a partir del estado actual de otra rama.
            \item Uno puede ir cambiando entre ramas a gusto.
            \item Una vez implementada la nueva funcionalidad, se hace un \alert{merge}.
                \begin{itemize}
                    \item Se ``fusiona'' una sub-rama con su rama padre.
                    \item Ojo! A veces pueden surgir conflictos.
                \end{itemize}
            \end{itemize}
        \item \alert{Pull}: bajar un estado del repositorio a la máquina local.
        \item \alert{Commit}: guardar los últimos cambios en la rama actual.
            \begin{itemize}
            \item Es como una foto que se le toma al código.
            \item Guarda los cambios localmente en tu máquina.
            \end{itemize}
    \end{itemize}
\end{frame}

\begin{frame}{Uso}{Glosario II}
    \begin{itemize}
        \item \alert{Stage}: prepara los archivos para un \textit{commit}.
        \item \alert{Clone}: clonar un repositorio a una máquina local.
        \item \alert{Push}: Subir los \textit{commits} al repositorio.
        \item \alert{Stash}: Es como un commit temporal para guardar los cambios.
        \item \alert{Fork}: Crear un repositorio en base a otro a partir de cierto \textit{commit}.
    \end{itemize}
\end{frame}

\subsection{¿Dónde aprender?}
\begin{frame}{¿Dónde aprender?}
    \begin{itemize}
        \item Ejemplo en ayudantía.
        \pause
        \item Recursos para estudio personal:
        \begin{itemize}
            \item \url{https://try.github.io}
            \item \url{https://training.github.com/kit/downloads/github-git-cheat-sheet.pdf}
        \end{itemize}
    \end{itemize}
\end{frame}

% Inicio de sección de Jaime

\section{Módulos}
\begin{frame}{Módulos}{Motivación}
  \begin{itemize}
    \item En Introducción a la Programación se suele escribir todo el código en un mismo archivo. \alert{Esto no es viable cuando nuestros programas empiezan a crecer.}
          \pause
          \begin{itemize}
            \item El mantenimiento del código es difícil
                  \pause
            \item Es casi imposible trabajar en equipo
                  \pause
            \item Se ve horrible
          \end{itemize}
          \pause
    \item Queremos tener una forma de poder dividir nuestro programa en varios archivos: \pause \alert{módulos}
            
  \end{itemize}
   
\end{frame}

\begin{frame}{Módulos}{Definición}
  \begin{definition}
    Un \alert{módulo} es un archivo que contiene definiciones y declaraciones de Python. \alert{El nombre del archivo es el nombre del módulo.}
  \end{definition}
  \pause
  \begin{itemize}
    \item Básicamente cualquier archivo escrito en Python constituye un módulo.
    \item Python por defecto ya trae varios implementados
  \end{itemize}
  \pause
  \begin{example}
    Un archivo llamado fibonacci.py constituye un módulo llamado fibonacci
  \end{example}
\end{frame}

\begin{frame}{Módulos}{Más sobre módulos}
  \begin{block}{Ejercicio}
      Descargar el archivo \texttt{modulo\_ayudantia.py} desde repositorio GitHub del curso.
  \end{block}
  
  \pause
  \begin{itemize}
    \item ¿Qué puede contener un módulo?
          \pause
          \begin{itemize}
            \item Variables
            \item Métodos
            \item Clases
          \end{itemize}
    \item ¿Qué puedo hacer con un módulo?
          \pause
          \begin{itemize}
            \item Usarlo en otro programa o módulo
          \end{itemize}
            
  \end{itemize}
    
\end{frame}

\section{Import}
\begin{frame}[fragile]{Import}{Como usar los módulos}
  Para importar un módulo en nuestro programa, escribimos en el principio de él:\\
  \begin{lstlisting}[language=Python]
  # Sentencias import
  import nombre_modulo
  # ...
  \end{lstlisting}
  Luego para usar algo del módulo importado:
  \begin{lstlisting}[language=Python]
  # ...
  nuevo_objeto = nombre_modulo.ClaseContenida()
  variable = nombre_modulo.variable_contenida
  nombre_modulo.funcion_contenida()
  \end{lstlisting}
  
  \begin{itemize}
      \item El nombre del módulo importado se agrega al namespace del script
  \end{itemize}
\end{frame}

\begin{frame}{Import}{Como usar los módulos}
    \begin{block}{Ejercicio}
    Abrir el intérprete de línea de comandos de Python en la carpeta donde se descargó \texttt{modulo\_ayudantia.py} e importarlo. Luego pruebe a usar sus funciones y clases.
    \end{block}
\end{frame}

\begin{frame}[fragile]{Import}{Otra forma de importar módulos}
  Existe otra forma de importar un módulo:
  \begin{lstlisting}[language=Python]
  # Sentencias import
  from nombre_modulo import funcion_contenida, ClaseContenida, variable_contenida
  # ...
  \end{lstlisting}
  
  Para ocupar lo que importamos:
  \begin{lstlisting}[language=Python]
  # ...
  nuevo_objeto = ClaseContenida()
  variable = variable_contenida
  funcion_contenida()
  \end{lstlisting}
    
  \begin{itemize}
    \item El nombre del módulo importado no queda en el namespace del script, sino que quedan las cosas que queremos importar directamente.
    \item Debemos colocar explícitamente todo lo que queremos traer del otro módulo
  \end{itemize}
  
\end{frame}

\begin{frame}{Import}{Otra forma de importar módulos}
    \begin{block}{Ejercicio}
        Abrir el intérprete de línea de comandos de Python en la carpeta donde se descargó \texttt{modulo\_ayudantia.py} e importarlo, \textbf{pero ahora de esta forma}. Luego pruebe a usar sus funciones y clases.
    \end{block}
    
    \begin{block}{Ejercicio}
        Crear un archivo llamado \texttt{math.py} con un único método llamado \texttt{hola\_mundo()}, que deberá imprimir en pantalla "Hola Mundo". Luego abra el intérprete de Python en la carpeta donde guardó el módulo, impórtelo y llame a la función que creó.
    \end{block}
    
    \pause
    \begin{itemize}
        \item ¿Funcionó?
    \end{itemize}
\end{frame}

\begin{frame}{Import}{Remarks}
  \textbf{Ojo:} Python buscará el módulo en el siguiente orden:
  \begin{itemize}
    \item Módulo integrado con Python
    \item Archivo nombre\_modulo.py ubicado en la misma carpeta del archivo del que se importa
    \item Archivo nombre\_modulo.py en el directorio de instalación.
  \end{itemize}
\end{frame}

\begin{frame}[fragile]{Import}{Haciendo trampa}
  Podemos incorporar al namespace todo lo que contiene otro módulo sin tener que nombrar las cosas de a una:
  \begin{lstlisting}[language=Python]
  from nombre_modulo import *
  \end{lstlisting}
  Para usar el contenido de ese módulo se hace igual que antes.
  \pause
  \begin{block}{Advertencia}
    \alert{Esto se considera en extremo una mala práctica}.\\
    \begin{center}
      \includegraphics[height=2cm]{Images/hay-tabla.jpg}
    \end{center}
  \end{block}
\end{frame}

\begin{frame}[fragile]{Import}{Main de un módulo}
    \begin{itemize}
        \item Cuando importamos un módulo, se ejecuta todo lo que no esté declarado bajo una función o clase.
        \item Queremos que esto suceda sólo si lo llamamos directamente (\texttt{python nombre\_modulo.py})
        \pause
        \item Solución: Agregar un condicional al módulo que contiene todo el código ejecutable
    \end{itemize}
    \begin{lstlisting}[language=Python]
if __name__ == "__main__":
    # Todo lo que quiera ejecutar si se llama al modulo.py
  \end{lstlisting}
\end{frame}

% Término de sección de Jaime


%% **********************
%% Sumario **************
%% **********************
% Al ponerle un * a \section hace que no aparezca en el índice.
\section*{Final Remarks}

\begin{frame}{Final Remarks}
  \begin{itemize}
    \item Usaremos GIT para entregar las tareas. \alert{Detalles: Próximamente}
    \item No escriban todo su código en un único archivo. Usen los módulos
    \item Jamás ocupen \texttt{from modulo import *}. Se irán al infierno si lo hacen.
    \item Revisen como ocupar el respositorio del curso y como plantear preguntas.
    \item Bonus: Si le envían un correo a un ayudante o a los  profesores, asegúrese de que el asunto cumpla con el patrón \texttt{IIC2233 - Mi Asunto}. Esto es para poder clasificar nuestros mensajes. Pero el foro es de preferencia, pues tu pregunta puede ayudar a los demás.
  \end{itemize}
\end{frame}


%% **********************
%% Lecturas o material ***
%% **********************
% Icons: http://tex.stackexchange.com/questions/68080/beamer-bibliography-icon
\appendix
\section<presentation>*{\appendixname}
\subsection<presentation>*{Lecturas y materiales}

\begin{frame}[allowframebreaks]
  \frametitle<presentation>{Lecturas y materiales}
      
  \begin{thebibliography}{10}
        
    \setbeamertemplate{bibliography item}[online]
    \bibitem{Doc}
    Última versión de Git
    \newblock {\em \url{http://git-scm.com/}}.
    \newblock 2015-03-06.
    
    \bibitem{Doc}
    Try-Git, por CodeSchool y Github.
    \newblock {\em \url{https://try.github.io}}.
    
    \bibitem{Doc}
    PEP 8 - Style Guide for Python Code
    \newblock {\em \url{https://www.python.org/dev/peps/pep-0008/}}.
    \newblock 01-Aug-2013.
  \end{thebibliography}
\end{frame}

\end{document}

\documentclass[10pt]{extarticle}

%Paquetes utilizados en esta tarea
\usepackage{logo}
\usepackage{fullpage}
\usepackage[utf8]{inputenc}
\usepackage[spanish]{babel}
\usepackage{epsfig}
\usepackage{amsmath}
\usepackage{amssymb}
\usepackage{epstopdf}
\usepackage[hidelinks]{hyperref}
\usepackage{algorithmic}
\usepackage[nothing]{algorithm}
\usepackage{graphicx}

%Definiciones de comandos, para reutilizar secuencias frecuentes o ahorrar
% c�digo
\newcommand{\RR}{\mathbb{R}}
\newcommand{\lb}{\\~\\}
\newcommand{\la}{\leftarrow}

\newcommand{\twopartdef}[4]
{
	\left\{
		\begin{array}{ll}
			#1 &  \text{si }#2 \\
			#3 &  \text{si }#4
		\end{array}
	\right.
}

\newcommand{\threepartdef}[6]
{
	\left\{
		\begin{array}{ll}
			#1 &  \text{si }#2 \\
			#3 &  \text{si }#4 \\
			#5 &  \text{si }#6
		\end{array}
	\right.
}

\makeatletter

\makeatother

\begin{document}

\begin{tabular}{ccl}
\begin{tabular}{c}
\psfig{file=puclogo.eps}
\end{tabular}
&\ \ \ & 
\begin{tabular}{l}
PONTIFICIA UNIVERSIDAD CATÓLICA DE CHILE\\
ESCUELA DE INGENIERÍA\\
DEPARTAMENTO DE CIENCIA DE LA COMPUTACIÓN
\end{tabular}
\end{tabular}

\begin{center}
\bf IIC2233 - Programación Avanzada\\
\bf 1° semestre 2015\lb

\vspace{0.5cm}

\bf {\Huge Actividad 22}
\end{center}

\section*{Threading}

\subsection*{Instrucciones}
\textit{ALERTA ALERTA!!!!!!} Godzilla ha llegado a San Joaquín!!!!! Es por esto que los soldados de San Joaquín necesitan tu ayuda para simular la pelea contra Godzilla!!, para ver si arrancar o pelear contra él. \\

Para esto los soldados de San Joaquín nos han entregado un informe con las especificaciones que se deben cumplir. Estas especificaciones se presentan a continuación: \\


\subsection*{Requerimientos}

La simulación debe:

\begin{itemize}
	\item
	Tener un Godzilla y varios soldados.
	\item
	Cada Guerrero tiene una velocidad de ataque, HP (vida), y ataque (daño)
	\item
	El Godzilla ataca de dos maneras. Con un \textit{pasivo} (ataque cada 8 segundos) que afecta a todos los soldado, restándole 3 de HP a cada uno de los guerreros. Además, cada vez que un soldado lo ataca, Godzilla devuelve un cuarto del ataque al soldado.
 
	\item
	La velocidad con la que atacan los Guerreros es aleatoria (Entre 4 y 19 segundos)
\end{itemize}

Como \textbf{BONUS}, la simulación debe:

\begin{itemize}
	\item
	Crear nuevos soldados cada x cantidad de segundos (x queda a su criterio) y agregarlos a la batalla
\end{itemize}

\subsection*{Notas}

\begin{itemize}
    \item 
    Deben trabajar en el main entregado
	\item
	Godzilla y cada Soldado deben existir en un thread independiente.

\end{itemize}

\subsection*{To - Do}

\begin{itemize}
	\item
	\textbf{(1.00 pts)} Terminar las clases pertinentes
	\item
	\textbf{(2.00 pts)} Simulación Correcta
	\item
	\textbf{(2.00 pts)} Threads funcionan correctamente
	\item
	\textbf{(1.00 pts)} Resultado con sentido
	\item
	\textbf{BONUS (1.0 pt)} 
\end{itemize}

\begin{figure}[ht!]
\centering
\includegraphics[width=90mm]{Gozilla.jpg}
\caption{RAAAWRR!! \label{overflow}}
\end{figure}

Fuente imagen: https://s-media-cache-ak0.pinimg.com/originals/bd/de/57/bdde57f44cb91f4189f39101f05d3ca6.jpg

\end{document}
